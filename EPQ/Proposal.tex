\documentclass[12pt]{article}
\usepackage[margin=1in]{geometry}
\usepackage{amsmath}
\usepackage{setspace}
\usepackage{parskip}

\title{EPQ Proposal}
\author{Candidate Name: YOUR NAME HERE\\Centre Number: XXXXX\\Candidate Number: XXXXX}
\date{}

\begin{document}

\maketitle

\section*{Working Title}
\textbf{Understanding the Difficulty of P vs NP: A Study of Relativization, Natural Proofs, and Algebrization}

\section*{Aims and Objectives}
The aim of this project is to understand why the P vs NP problem remains unsolved, despite being one of the most studied open problems in theoretical computer science and mathematics. I will focus on three major theoretical frameworks—relativization, natural proofs, and algebrization—that have been shown to act as barriers to current proof techniques. My objectives are to:
\begin{itemize}
    \item Define and explain the classes P, NP, and NP-completeness.
    \item Introduce and explain each of the three major barriers in detail.
    \item Evaluate their impact on attempts to resolve the P vs NP problem.
    \item Reflect on what these barriers suggest about the future direction of complexity theory.
\end{itemize}

\section*{Rationale}
I have chosen this topic because I am applying to study Computer Science at the University of Cambridge, and I want to demonstrate genuine engagement with the kind of deep theoretical thinking that the course requires. This topic allows me to explore advanced concepts in computational complexity, logic, and meta-mathematics, while producing a structured and focused research project. Rather than speculating on a solution to P vs NP, I will analyse why a solution has proved so elusive, which is a far more realistic and academically rigorous goal for an EPQ.

\section*{Research Methods and Approach}
My research will consist of a structured theoretical investigation using primary and secondary sources. I will:
\begin{itemize}
    \item Read and interpret foundational academic papers in complexity theory.
    \item Consult textbooks and lectures to support my understanding of complex proofs.
    \item Organise my findings into thematic chapters corresponding to each barrier.
    \item Include critical commentary and comparisons between approaches.
\end{itemize}
This will be a literature-based project with analytical commentary, not an experimental or practical investigation.

\section*{Resources}
\begin{itemize}
    \item \textit{Computational Complexity: A Modern Approach} by Arora and Barak
    \item \textit{Introduction to the Theory of Computation} by Michael Sipser
    \item “Relativizations of the P=?NP Question” by Baker, Gill, and Solovay (1975)
    \item “Natural Proofs” by Razborov and Rudich (1994)
    \item “Algebrization: A New Barrier in Complexity Theory” by Aaronson and Wigderson (2008)
    \item Blog posts and lecture materials by Scott Aaronson and Ryan O'Donnell
    \item Selected undergraduate-level lecture notes from MIT and CMU
\end{itemize}

\section*{Expected Outcomes}
By the end of the project, I will:
\begin{itemize}
    \item Produce a 5,000–6,000 word report suitable for a mathematically literate audience.
    \item Demonstrate deep understanding of theoretical computer science topics.
    \item Critically assess the limits of current mathematical techniques in resolving the P vs NP question.
    \item Reflect on how these limitations shape the future of computational complexity.
\end{itemize}

\end{document}
